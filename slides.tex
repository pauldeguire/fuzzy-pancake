\documentclass{beamer}
%% Possible paper sizes: a0, a0b, a1, a2, a3, a4.
%% Possible orientations: portrait, landscape
%% Font sizes can be changed using the scale option.

\usetheme{Boadilla}
\usecolortheme{beaver}
\beamertemplatenavigationsymbolsempty
% \usecolortheme{Entrepreneur}
% \usecolortheme{ConspiciousCreep}  %% VERY garish.
% Themes by Lian Tze Lim found here: https://www.overleaf.com/latex/templates/yet-another-beamerposter-theme-with-variable-sizes-and-colour-themes-portrait-version/jgzfyrgpmvgk
% colourthemes from \url{http://colourlovers.com}

% This file provides examples of some useful macros for typesetting
% dissertations.  None of the macros defined here are necessary beyond
% for the template documentation, so feel free to change, remove, and add
% your own definitions.
%
% We recommend that you define macros to separate the semantics
% of the things you write from how they are presented.  For example,
% you'll see definitions below for a macro \file{}: by using
% \file{} consistently in the text, we can change how filenames
% are typeset simply by changing the definition of \file{} in
% this file.
% 
%% The following is a directive for TeXShop to indicate the main file
%%!TEX root = diss.tex

\usepackage{amsmath}

\newcommand{\NA}{\textsc{n/a}}	% for "not applicable"
\newcommand{\eg}{e.g.,\ }	% proper form of examples (\eg a, b, c)
\newcommand{\ie}{i.e.,\ }	% proper form for that is (\ie a, b, c)
\newcommand{\etal}{\emph{et al}}

% Some useful macros for typesetting terms.
\newcommand{\file}[1]{\texttt{#1}}
\newcommand{\class}[1]{\texttt{#1}}
\newcommand{\latexpackage}[1]{\href{http://www.ctan.org/macros/latex/contrib/#1}{\texttt{#1}}}
\newcommand{\latexmiscpackage}[1]{\href{http://www.ctan.org/macros/latex/contrib/misc/#1.sty}{\texttt{#1}}}
\newcommand{\env}[1]{\texttt{#1}}
\newcommand{\BibTeX}{Bib\TeX}

% Define a command \doi{} to typeset a digital object identifier (DOI).
% Note: if the following definition raise an error, then you likely
% have an ancient version of url.sty.  Either find a more recent version
% (3.1 or later work fine) and simply copy it into this directory,  or
% comment out the following two lines and uncomment the third.
\DeclareUrlCommand\DOI{}
\newcommand{\doi}[1]{\href{http://dx.doi.org/#1}{\DOI{doi:#1}}}
%\newcommand{\doi}[1]{\href{http://dx.doi.org/#1}{doi:#1}}

% Useful macro to reference an online document with a hyperlink
% as well with the URL explicitly listed in a footnote
% #1: the URL
% #2: the anchoring text
\newcommand{\webref}[2]{\href{#1}{#2}\footnote{\url{#1}}}

% epigraph is a nice environment for typesetting quotations
\makeatletter
\newenvironment{epigraph}{%
	\begin{flushright}
	\begin{minipage}{\columnwidth-0.75in}
	\begin{flushright}
	\@ifundefined{singlespacing}{}{\singlespacing}%
    }{
	\end{flushright}
	\end{minipage}
	\end{flushright}}
\makeatother

% \FIXME{} is a useful macro for noting things needing to be changed.
% The following definition will also output a warning to the console
\newcommand{\FIXME}[1]{\typeout{**FIXME** #1}\textbf{[FIXME: #1]}}

\newcommand{\stretchabs}[1]{\ensuremath{\left\lvert#1\right\rvert}}
\newcommand{\abs}[1]{\ensuremath{\lvert#1\rvert}}
\newcommand{\sq}[0]{\ensuremath{^{\textrm{2}}}}
\newcommand{\infinity}[0]{\ensuremath{\infty}}

% Use this one in the middle of sentences:
\newcommand{\superbkerned}{Super\kern-.1em\textit{B}\kern.2em}
% Use this one at the end of a sentence.
\newcommand{\superb}{Super\kern-.1em\textit{B}}

\DeclareMathOperator{\sech}{sech}
\DeclareMathOperator{\csch}{csch}
\DeclareMathOperator{\arcsec}{arcsec}
\DeclareMathOperator{\arccot}{arcCot}
\DeclareMathOperator{\arccsc}{arcCsc}
\DeclareMathOperator{\arccosh}{arcCosh}
\DeclareMathOperator{\arcsinh}{arcsinh}
\DeclareMathOperator{\arctanh}{arctanh}
\DeclareMathOperator{\arcsech}{arcsech}
\DeclareMathOperator{\arccsch}{arcCsch}
\DeclareMathOperator{\arccoth}{arcCoth} 
\DeclareMathOperator{\cov}{cov}

\newcommand{\aref}[1]{Appendix~\ref{#1}}
\newcommand{\exclude}[1]{}

% END


\usepackage{hyperref}
\usepackage{color}
\definecolor{mymauve}{rgb}{0.58,0,0.82}

\usepackage{units}
\usepackage{listings}
%\usepackage[format=hang,justification=centering]{caption}
\usepackage[utf8]{inputenc}
\usepackage[T1]{fontenc}
%\usepackage{libertine}
%\usepackage[scaled=0.92]{inconsolata}
%\usepackage[libertine]{newtxmath}

\renewcommand{\figurename}{}

\author[J.-F. Caron]{Jean-Fran\c{c}ois Caron}
\title{PyROOT in the Lab}
\institute{Queen's University}
% Optional foot image
%\footimage{\includegraphics[height=1.6cm]{figures/UH_Red}}

\begin{document}
\lstset{language=Python, showstringspaces=false, commentstyle=\ttfamily}
\lstset{keywordstyle=\color{blue},commentstyle=\color{brown},identifierstyle=\color{black}}
\lstset{stringstyle=\color{teal}, basicstyle=\ttfamily\tiny}
\begin{frame}
\titlepage
\end{frame}

\begin{frame}\frametitle{What is PyROOT?}
PyROOT is a bridge allowing you to call C++ ROOT functions from a python program. It is automatically generated from the ROOT source code, so the classes and functions are all equivalent.

\begin{columns}[T]
\begin{column}{0.5\textwidth}
Pros
\begin{itemize}
\item Few new interfaces to learn.
\item High-performance with built-in ROOT objects.
\item Flexibility and scope of Python language and standard library.
\item Can add-in 3rd-party numerical python libraries.
\end{itemize} 
\end{column}

\begin{column}{0.5\textwidth}
Cons
\begin{itemize}
\item Python-side performance can be worse than doing it in C++.
\item Need to code-switch between Python and C++.
\item More libraries to install.
\item Sometimes need workarounds for ROOT weirdness.
\end{itemize}
\end{column}
\end{columns}
\vspace{0.2cm}

My strategy: use compiled C++ code with ROOT libraries for heavy number-crunching, but use PyROOT for exploration, interactive use, and plotting. 

\end{frame}

\begin{frame}\frametitle{Your Primary Tool: TGraph}
The ROOT TGraph is a basic 2D graph of $X$ vs $Y$.
\lstinputlisting[language=Python]{demo_tgraph.py}
Numpy arrays can also be used instead of array.array.\\
You can also create an empty \lstinline$ROOT.TGraph(N)$ and fill the points in one by one using \lstinline$g.SetPoint(i,a,b)$.

\small{
Exercises (3 minutes):
\begin{enumerate}
\item Try plotting \lstinline$ROOT.TMath.Sin$ or your favourite function.
\item Try \lstinline$g.GetXaxis().SetTitle("foo")$ and \lstinline$g.SetTitle("bar")$.
\end{enumerate}
}
\label{demo_tgraph}
\end{frame}

\begin{frame}\frametitle{Histograms: TH1D}
Histograms are frequently used in particle physics. In ROOT they play a more central role than even TGraphs.  You create them with a certain range of bins and fill them with values.

\lstinputlisting[language=Python]{demo_th1d.py}

The TH1D class is full-featured: you can set variable bin widths, fill with different weights, change bin statistics, interface to fitting, etc.

\tiny{Note: there is no reason to use the other TH1* types.}
\end{frame}

\begin{frame}\frametitle{Reading Files 0: Fake Data}
So you can make graphs, but how do you get the data from a file into the program?
First, let's generate an example text file to work with.

\lstinputlisting[language=Python]{demo_generate_data.py}
Exercise (1 minute): Examine the output file with \lstinline$wc$ and \lstinline$less$.
\label{generate}
\end{frame}

\begin{frame}\frametitle{Reading Files 1: Manually Filling a TTree}
\lstinputlisting[language=Python]{demo6_ttree_manual.py}
\end{frame}

\begin{frame}\frametitle{Reading Files 2: TTree.ReadFile}
\lstinputlisting[language=Python]{demo2_ttree_readfile.py}
\label{convert}
\end{frame}

% \begin{frame}\frametitle{Reading Files 2: numpy.loadtxt}
% Pros: Can use ``slicing'' syntax and other numpy goodness. Can use optional converter argument.

% Cons: No built-in way to save to optimized file, still need a plotting system.
% \lstinputlisting[language=Python]{demo3_numpy_loadtxt.py}

% After this you would manipulate the data to fill numpy arrays with the quantities to be graphed like in slide \ref{demo_tgraph}.
% \end{frame}

% \begin{frame}\frametitle{Reading Files 3: Built-in CSV module}
% Pros: More structured than reading a plain file. Can handle Excel/spreadsheet files. Very flexible since you just read strings.

% Cons: Slower than numpy.loadtxt. Manual conversion to numbers.
% \lstinputlisting[language=Python]{demo5_csv.py}

% Inside the loop you would do processing or fill arrays to graph with ROOT.  You can also parse the \# comments to read parameters like tau and the random seed.

% % You could also read the file normally using python's normal file objects, but this is error-prone; I would not recommend it:
% % \lstinputlisting[language=Python]{demo4_python_file.py}
% \end{frame}

% \begin{frame}\frametitle{Reading Data Files 4}
% So which way should you use?

% Any of them, depending on your needs and the format of your input file.  If you have a large file ($ > \unit[10]{MB}$), I would highly recommend making a TTree inside a TFile for the speed gains.  

% If your input file is difficult to work with (\ie TTree::ReadFile won't work), you will need to fill the tree manually.
% \end{frame}

\begin{frame}\frametitle{Working with TTrees}
Exercise (5 minutes): enter these commands interactively in python.
Remember to use tab-completion to save on typing!
\lstinputlisting[language=Python]{demo7_work_trees.py}
\end{frame}

\begin{frame}\frametitle{Making Plots}
% TH*, TGraph(Errors), THStack and TMultiGraph
% How to make from TTree or from arrays.
% Colour, axes, legends
\lstinputlisting[language=Python]{demo8_ttree_draw.py}
Exercise (3 minutes):
\begin{enumerate}
\item Draw a histogram or graph of a quantity of your choosing, with proper axis labels.
\end{enumerate}
\end{frame}

\begin{frame}\frametitle{Saving and Exporting}
% TFile, output formats: tex, pdf, png
To properly save figures, you need to save the TCanvas, not the TGraph or TH1.  The active canvas can be saved with \lstinline$ROOT.gPad.SaveAs("foo.pdf")$ or:
\lstinputlisting[language=Python]{demo9_saving.py}
Exercise (2 minutes):
\begin{enumerate}
\item Save the figure from the last exercise in the four formats shown.  
\item Try to view the ouput files (as text or as figures)\\
(\lstinline$evince$ for pdf, \lstinline$eog$ for png).
\item Look at the size difference in the output files with \lstinline$ls -lh plots$.
\end{enumerate}
\label{saving}
\end{frame}

\begin{frame}\frametitle{Aside: The Power of PyROOT}
\small{
You can add all sorts of functionality in Python.  Here is a function I made to automatically timestamp and move a file.  I use it before saving figures.
\lstinputlisting[language=Python]{ArchiveExisting.py}
You could write this in C++ too, but in Python it's way easier!\\
Homework: Use the python documentation to figure out exactly how this works.
}
\end{frame}

\begin{frame}\frametitle{Advanced Drawing}
\lstinputlisting[language=Python]{demo9b_advanced_drawing.py}
\small{This method can take you surprisingly far, and it's very fast because the looping happens outside of Python.}
\end{frame}

\begin{frame}\frametitle{Stacking Histograms}
\lstinputlisting[language=Python]{demo10_stacks.py}
\label{thstack}
\end{frame}

\begin{frame}\frametitle{Stacking Graphs}
\lstinputlisting[language=Python]{demo11_stacks.py}
\end{frame}


\begin{frame}\frametitle{Error Bars}
Use TGraphErrors, TGraphAsymmErrors, etc.
\lstinputlisting[language=Python]{demo12_errors.py}
Unfortunately you cannot create TGraphErrors directly with TTree.Draw.
Exercise (0.5 minutes):
\begin{enumerate}
\item Try \lstinline$g.Print()$.
\end{enumerate}

\end{frame}

\begin{frame}\frametitle{Aside: The Power of PyROOT 2}
\lstinputlisting[language=Python]{AddErrors.py}
\end{frame}

\begin{frame}\frametitle{Basic Fitting 1}
ROOT has too many ways to fit things.  This is just one way.
\lstinputlisting[language=Python]{demo13_fit.py}
Note the same process works for TGraphs.
Exercise (1 minute):
\begin{enumerate}
\item Do \lstinline$ftr.$ and press tab twice.
\item Try some of the interesting methods listed.
\end{enumerate}

\label{fitting1}
\end{frame}

\begin{frame}\frametitle{Basic Fitting 2}
You can define your own function with a TF1 or with a C++ function.
\lstinputlisting[language=Python]{demo14_fit2.py}
\end{frame}

\begin{frame}\frametitle{Compiled C++ Extensions}
\lstinputlisting[language=C]{demo15_cpp_extension.C}
\lstinputlisting[language=Python]{demo15_usage.py}
Exercise (3 minutes):
\begin{enumerate}
\item Run \lstinline$demo15_usage.py$ with \lstinline$ipython -i$
\item Use the \lstinline$\%timeit$ "magic" to compare the speed of C++ and Python.
\end{enumerate}
\end{frame}

\begin{frame}\frametitle{Neglected Topics}
% TFile, output formats: tex, pdf, png
\begin{enumerate}
\item ``Collection'' objects in TTrees
\item 3D and higher plots, profile plots
\item TDataFrame
\item ``Out parameters''
\item TTree.Scan
\end{enumerate}
\end{frame}

\begin{frame}\frametitle{Getting Help}
% ROOT Forum, python IRC, C++ IRC
% Reading reference pages
% Reading source code
\begin{enumerate}
\item The ROOT forum is very active: \url{https://root-forum.cern.ch/}
\item ROOT User's Guide: \url{https://root.cern.ch/root/htmldoc/guides/users-guide/ROOTUsersGuide.html}
\item ROOT Reference Guide: \url{https://root.cern/doc/master/}
\item PyROOT-specific tutorials:\\ \small{\url{https://root.cern/doc/master/group__tutorial__pyroot.html}}
\item The FreeNode IRC channels \lstinline$\#python$ and \lstinline$\#c++-basic$ are helpful.
\item Python official documentation: \url{https://docs.python.org/3/}
\end{enumerate}
\end{frame}

\begin{frame}\frametitle{Major Exercise/Homework}
You will probably not finish (or start?) this during the workstop time.
\small{
\begin{enumerate}
\item Generate a new data set (as on slide \ref{generate}) with the binomial distribution for\\ N = 20, 200, and 2000 trials.
\item Convert this data set to a ROOT TFile with a TTree in it (as on slide \ref{convert}).
\item Plot the distributions in TH1Ds and put them together in a THStack \\(as on slide \ref{thstack}).
\item Fit each of the distributions with a Gaussian function, note the Chi2/NDf (as on slide \ref{fitting1}).
\item Save the produced plot in your favourite format (as on slide \ref{saving}).
\item Bonus: put the fit results in a TPaveText on top of the THStack plot.
\end{enumerate}
}
\end{frame}

\end{document}